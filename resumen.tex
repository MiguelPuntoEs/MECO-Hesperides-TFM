\section*{Resumen}
Las diversas políticas monetarias adoptadas por los principales bancos centrales tuvieron efectos diferentes sobre la inflación durante la pandemia de COVID-19 y, especialmente, después de ella. Mientras que algunos países como Suiza y Japón registraron tasas de inflación anuales moderadas por debajo del 3,5\% (medidas por sus respectivos Índices de Precios al Consumo), no ocurrió lo mismo en otras zonas monetarias como Estados Unidos, la zona euro o el Reino Unido.

En los últimos años, en el mundo académico se han utilizado enfoques teóricos para explicar los orígenes de la inflación, entre ellos la teoría cuantitativa del dinero, el nuevo marco keynesiano, la teoría monetaria moderna y la teoría fiscal del nivel de precios. Dado que los principales bancos de las áreas monetarias y países mencionados aplicaron políticas con notables diferencias en términos de agregados monetarios amplios correlacionadas con resultados de inflación diversos, la teoría cuantitativa \enquote{amplia} del dinero puede ser un marco teórico adecuado para analizar el efecto de los agregados monetarios amplios sobre la inflación.

Por lo tanto, un análisis empírico como el que se presentará en este trabajo puede llevar a conclusiones relevantes sobre la posibilidad de utilizar las variaciones de los agregados monetarios y de la velocidad del dinero para establecer un vínculo con la inflación. Se utiliza un modelo de cambio de régimen (\textit{Markov-switching model}) para comprobar el impacto de las magnitudes monetarias (cambios en la oferta monetaria y en la velocidad del dinero) sobre la inflación para Suiza, Japón, Estados Unidos, la Eurozona y el Reino Unido.

El hecho de que se utilicen distintas zonas monetarias para el presente análisis permite un estudio multi-regional y multidivisa de las relaciones entre los agregados monetarios, la velocidad del dinero y la inflación.

Del análisis realizado para todas las áreas monetarias mencionadas se derivan principalmente dos conclusiones relevantes: primero, los cambios en la velocidad de circulación del dinero son estacionarios presentando un valor medio negativo y relativamente uniforme entre estas, con una variación similar; segundo, los cambios en la oferta monetaria en sentido amplio presentes y pasados son significativos a la hora de explicar cambios en el nivel de precios de las diferentes áreas monetarias.