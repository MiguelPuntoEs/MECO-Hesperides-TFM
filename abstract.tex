\begin{otherlanguage}{english}
  \section*{Abstract}
  Diverse monetary policies taken by leading central banks did have different effects on inflation during and especially after the COVID-19 pandemic. While certain countries such as Switzerland and Japan registered moderate annual inflation rates under 3.5\% (as measured by their respective Consumer Price Index), that was not the case for other monetary areas such as the United States, the Eurozone, or the United Kingdom.

  In the last years, theoretical approaches have been used in academia to explain the origins of inflation, including the quantity theory of money, the new Keynesian framework, the modern monetary theory, and the fiscal theory of the price level. Since the leading banks from the aforementioned monetary areas and countries implemented policies with remarkable differences in terms of broad money aggregates correlating with diverse inflation results, the \enquote{broad} quantitative theory of money can be a suitable theoretical framework to analyze the effect of broad monetary aggregates on inflation.

  Therefore, an empirical analysis such as the one to be presented in this publication can lead to relevant conclusions about the possibility of using changes in monetary aggregates and money velocity in order to establish a link to inflation. A regime-switching model (Markov-switching model) is used to test the impact of the monetary variables (changes in money quantity and money velocity) on inflation for Switzerland, Japan, the United States, the Eurozone, and the United Kingdom.

  The fact that different monetary areas are used for the present analysis allows for a multi-region, multi-currency study of relationships between monetary aggregates, money velocity, and inflation.

  From the analysis carried out for all the monetary areas mentioned above, two main conclusions are relevant: first, changes in the velocity of money circulation are stationary and have a negative and relatively uniform mean value among them, with a relatively similar variation among them; second, present and past changes in the broad money supply are significant in explaining changes in the price level of the different monetary areas.
\end{otherlanguage}